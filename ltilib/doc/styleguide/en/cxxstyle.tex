\newcommand{\staticcast}{\code{static\_cast}}

\chapter{C++ Programming Style Guide}

The Programming Style Guide provides several necessary rules to keep a clean
and consistent development environment.  Through this rules it is easier for a
developer to find his way in the code, even if he is not the original author.

The main goal of the \ltilib\ is to provide a system independent library,
which follows as much as possible the ANSI standards.  The system should work
on different \product{Unix}-Platforms and on \product{Windows}, but it is
specially maintained for \product{Linux} and \product{WindowsNT} machines.  To
achieve this, strict programming discipline is required.  System dependent
``tricks'' are not allowed.

In those cases where is not possible to write system independent code (for
example, due to direct hardware access), the platform dependent parts must be
isolated in the smallest possible units.  An access class have to be
specified, which provides access to all special supported operations.  No
direct platform dependent access is allowed.

\section{Organisation of the data}

\subsection{Version control}

\index{version control|see{CVS}} All code files will be administrated using a
\inIndex{CVS}-database.  The repository will be on the CVS-Server.  A few
hints to use CVS are available.  There are many graphic front-ends to simplify
the use of this version control system, like \product{WinCVS} for
\product{Windows} (\url{http://www.wincvs.org/}) or \product{Cervisia} for
\product{Linux/KDE} (\url{http://cervisia.sourceforge.net/}).

Each developer should not forget to comment its changes when committing into
the repository.  To avoid confusion, he/she should try to get only one copy of
the \ltilib\ (for the CVS-Server there is no problem, but some developers have
lose some work because they did changes somewhere else...).

\subsection{\inIndex{File conventions}}

\index{filenames|see{file conventions}}
\begin{itemize}
\item Header files have always the extention \code{.h}.
\item Source files have always the extention \code{.cpp}.
\item Different words in a file name are written together, and each word
  (except the first one) must begin with an uppercase letter.  The first word
  must be \code{lti}.  Examples: \code{ltiFunctor.h} \code{ltiLinearFilter.h}
\item Files which contain implementation for inline or template methods or
  functions must be clearly denoted with \code{\_template} or
  \code{\_inline}.  For example \code{ltiMatrix\_template.h}
  \index{\code{template}} (see also section \ref{ssec:templates}).
\end{itemize}

\section{Tools}

\subsection{Debugging}

Search your \emph{bugs} with a \inIndex{Debugger}.  On \product{Windows NT}
you can use the \visualc-Debugger.  On \product{Linux} there are many
front-ends for the GNU-Debugger gdb, like \product{xxgdb}, \product{kdbg} and
\product{ddd}.  A class must work without problems before it is integrated in
the rest of the \ltilib.  To test your new functors, you can use the
\code{tester} project provided with the library.

Other commercial products like \product{Purify} can help searching for memory
problems like memory leaks or outer-bounds access of arrays.

If a program crashes, it usually helps to catch the \code{lti::exception}s and
ask their content with the method \code{what()}) (see also section
\ref{sssec:exception}).  It also helps to check if the called \code{apply()}
or other functor methods return \code{false}, and in that case the cause of
the problem can be checked with the \code{getStatusString()} method.

\subsection{Optimization}

Try to implement efficient code, but in a way that it remains
maintainable.  You can use \product{gprof}, \product{valgrind} or the
\visualc\ profilers to detect the critical parts of your code, which
may require special attention by optimization.

\section{General programming conventions}

\subsection{Preamble}

Each violation to the programming conventions must be documented.  It is not
always possible to respect all conventions.  There are even cases where they
are contradictory and following one will violate another.  Each case has to
be studied separately to decide which solution makes more sense.  Discuss
these cases with other \ltilib\ developers.

\subsection{Name conventions}

The names of classes, methods, functions and variables must always make sense.
They all must be in English.  Comments must be also in English.  These way,
your code can be understand all around the world.

\section{C++-Programming}

C++ was chosen as the main language for the \ltilib\ because it is possible to
efficiently implement time and memory intensive algorithms (which are common
in computer vision) without sacrificing code maintainability and using modern
object oriented approaches.  It is also important to know where to look up for
information.  There are two helpful books \index{C++ Books}: \cite{primer} is
a very good introductory book for C++ and \cite{stroustrup}, in which almost
everything about C++ can be found, even if it is sometimes difficult to find.

\subsection{File organization}

\newcommand{\thelog}{Id}

\begin{itemize}
\item each file with source code begins with a comment including information
  about the file, like licence conditions, filename, authors, creation date,
  etc.  If a special comment is present (see \code{\$\thelog} in the next
  example), \cvs\ automatically enters some information in the file history:

{\small
\begin{alltt}
/*--------------------------------------------------------------
 * project ....: LTI Digital Image/Signal Processing Library
 * file .......: ltiThread.h
 * authors ....: Thomas Rusert
 * organization: LTI, RWTH Aachen
 * creation ...: 03.11.99
 * revisions ..: $\thelog: ltiThread.h,v 1.6 2003/02/17 07:17:01 author Exp $
 */
\end{alltt}
}

\item you should avoid multiple inclusion of header files:
  \index{\code{include}}
\begin{verbatim}
#ifndef HEADER_FILE_NAME_H
#define HEADER_FILE_NAME_H
...
#endif
\end{verbatim}

\item each class definition required in other header or implementation files,
  must have its own header file.
\item header files of the \ltilib\ must be included with
  \verb+#include "filename"+, system files with
  \verb+#include <filename>+.
\item cyclical \verb+#include+s have to be avoided.
\item each function has to show a comment documenting the interface and the
  functionality of the function or method.  The comment format must follow the
  Doxygen style.
\end{itemize}

\subsection{\inIndex{Name conventions}}
All \ltilib\ declarations must be done within the namespace
\index{\code{namespace}} \code{lti}, except those members which extend other
libraries (like \stl (STL)).

\begin{itemize}
\item names of data types, structs, classes, typedefs, enums, functions and
  methods begin with a lowercase letter.  Only the enum constants and the
  template parameters begin with an uppercase letter.\index{name!method}
  \index{name!class} \index{type definition} \index{enum} \index{name!function}
  \index{struct}
\item names with more than one word are written without separators, and each
  word (except the first one) must be written in uppercase.  For example
  \code{linearEquationSystemSolutionMethod}.  The only exception are the
  type names used in template classes to designate different properties of
  template parameters.  This can be done to keep some concordance with
  the STL.  For example in matrix<T>, the type of T can be accessed through
  the \code{value\_type} type.  In case of doubt, prefer the LTI-Lib standard 
  (thisWay instead of this\_way).  In other words, underscores should only
  be found a very few times in template classes.
\item names must not begin with an \inIndex{Underscore} (\code{\_}).
\item use names that makes sense (not \code{mIn} for \code{maxIndex})
\item use C++ comments \code{//}.  Comments used for documentation
  generation with \product{Doxygen}, must follow the Doxygen conventions, \ie\
  begin with \verb+/**+, end with \verb+*/+.  In documentation comments with
  multiple lines, each line must begin with '\code{*}'.  See also section
  \ref{sec:doku}.
\item names must sufficiently differ.  Differences in lowercase or
  uppercase letters only lead usually to confusion.
\end{itemize}

\subsection{Class declaration and definition}

\subsubsection{Visibility sections}
\label{sssec:sichtbarkeit}
\begin{itemize}
\item The order for the visibility sections in class definitions must be:
\begin{enumerate}
\item \code{public}
\item \code{protected}
\item \code{private}
\end{enumerate}
\item \code{friend} declarations are not allowed!  The only valid exception is
  if a class needs access to protected attributes of enclosed classes (like
  iterators).  Otherwise the class design must be revised.
\item methods should not be defined in the class declarations, \ie\ in the
  header file there should not be any code implementation.  Template and inline
  functions should be implemented in \code{\_inline.h}
  bzw. \code{\_template.h} files, which are appended using \code{\#include}.
  Due to a bug of \visualc (version 6.0), the implementation of enclosed
  classes of a template class or template methods must be done directly in the
  class declaration.  This is the only accepted exception until now.
  (see also section \ref{ssec:templates})
\end{itemize}

\subsubsection{Methods and global functions}
\begin{itemize}
\item global functions are not allowed (except for general tools like minimum,
  maximum or general mathematical functions like sin(), cos(), tan(), etc.)
\item a method that does not change the state of an object must be declared as
  \code{const}.  These are for example methods to access the value of some
  protected attributes.
\item classes with attributes must define a \inIndex{copy-constructor} and a
  copy member.  These receive as parameter a \code{const} reference
  to an object of the same type.  The documentation must specify, if these
  methods do a deep copy or a shallow copy (\ie\ if the class contains
  pointers, if the whole pointed data are copied, or only the pointers.)
  In the \ltilib\ almost every copy method and copy constructor do
  a deep copy.
\item classes that allocate data with \code{new} must define the operator `=',
  which also receives a const reference to an object of the same type.
\item methods and functions which do not change the contents of the
  parameters, must declare them as \code{const}.
\item classes with virtual functions and own attributes must declare the
  virtual method \code{clone()}, which have exactly the same interface than
  the \code{clone()} method of the parent class.  These method looks as
  follows:
\begin{verbatim}
rootClass* Foo::clone() const {
  return new Foo(*this);
}
\end{verbatim}
\item classes with virtual methods must have a virtual destructor.
\item a public method must not return references or pointers to local
  variables, even if these variable are declared \code{static}.  The reason
  for this restriction is that in multi-threaded programs this can cause
  crashes of unpredictable behaviours if parallel accesses takes place.
\item a public method must not return references or pointers to protected
  attributes, except if the returned values are declared \code{const}.
\item variable argument lists (...) in function definitions are not allowed,
  due to the lack of type checking possibilities.  You can overload methods
  to provide the desired functionality.
\item the names of the \inIndex{arguments} must be the same in the declaration
  and definition of functions and methods.  The names of the arguments must
  be self explanatory.
\item the returned value of a function or member must be given explicitely.
\end{itemize}

\subsubsection{Constants and variables}

\begin{itemize}
\item Declare constant values with \code{const} or \code{enum} instead of
\verb+#define+.
\item Do not use ``magic numbers'' in your code (\code{int value = 0x42}).
\item Declare your variables in the smallest possible scope (if this does not
  affect considerably the efficiency of your program!).
\item Initialize your variables before their first use (otherwise you will get
  megabytes of warnings from tools like \product{Purify} without any logical
  reason).
\item Never compare floating point types for equality with \code{==},
  this is also for comparing to zero, like \code{==0}. Instead use the
  functions \code{closeTo()} and \code{closeToZero()}, respectively,
  which compare the difference/value to a small number.
\item Due to the fact that pointer operations are always one of the most usual
  error sources, try to use references instead.  If you allocate some memory,
  check your code for memory leaks.
\end{itemize}

\subsection{Type casts}

\begin{itemize}
\item \inIndex{Type casts} in C-style are not allowed!(new=(NewType)old).
  C++-Casts must follow following conditions:
\begin{itemize}
\item with polymorphic classes, it can be necessary to cast a static type into
  a dynamic one.  This can be achieved using
  \verb+dynamic_cast+\index{\code{dynamic\_cast}}.  This way type checking
  will take place at compilation time and at run time, to ensure that the used
  types are compatible:
\begin{verbatim}
class A {
};

class B: public A {
};

class C {
};

A a1,*a2,*a3;
B b1,*b2;
C *c2;
a2=&b1;                   // a2 has dynamic type B
a3=&a1;                   // a3 has dynamic type A
b2=dynamic_cast<B*>(&a2); // Ok, notNull(b2)
c2=dynamic_cast<C*>(a2);  // Error at compile type:
                          // Types not compatible
b2=dynamic_cast<B*>(a3);  // b2 is Null!, because B is
                          // not the type of a3
\end{verbatim}
\item if you cannot avoid it (for example using some system functions) you can
  replace an old C-case with
\begin{verbatim}
new=reinterpret_cast<NewType>(old)
\end{verbatim}
This is though one of the most dangerous error sources. The use of the C++
cast emphasises when reading code, that the real type of a variable is just
being ignored.

C cast are not allowed due to the fact that they are not type secure, and the
\ltilib\ follows a strong type security philosophy.

\item the secure type conversion between static types, which cannot be
  implicitly casted, can be done in C++ with \code{static\_cast}.  The
  compiler can decide if a type conversion is allowed or not.  For example:
\begin{verbatim}
 float f;
 double d;

 d = f;                    // valid
 f = d;                    // not valid
 f = static_cast<float>(d);// valid
\end{verbatim}

\staticcast is also used in the \ltilib\ to solve conflicts between signed and
unsigned types.  For example:

\begin{verbatim}
  // compare the sizes of an lti::vector and a
  //  std::vector
  lti::vector ltivector(5);
  std::vector stdvector(6);

  if (ltivector.size() ==
      static_cast<int>(stdvector.size())) {
  ...
  }
\end{verbatim}

or if this comparison is not in a time critical section of your code, this
slower version can be used:
\begin{verbatim}
  // compare the sizes of an lti::vector and a
  // std::vector
  lti::vector ltivector(5);
  std::vector stdvector(6);

  // use the integer constructor to convert the
  // unsigned type into a signed int
  if (ltivector.size() == int(stdvector.size())) {
  ...
  }
\end{verbatim}

\item the last example shows the other C++ alternative to cast types using
  constructors.  For example:
\begin{verbatim}
float f;
int i;

i=5;
f = float(i);  // float constructor receives an
               // int parameter
\end{verbatim}
\end{itemize}

This possibility must be considered carefully, because it can be very slow depending on the used
compiler.
\item use 0 instead of NULL. To check pointers if they are 0 or not, use the
  predicates \code{isNull} or \code{notNull}.
\item Never ever use \verb+const_cast+.  This is confusing due to the
  fact that something declared \code{const} must remain const, and
  should not be changed by the code. Exceptions exist when calling
  functions in other languages (e.g. LAPACK) that assure constness of
  a pointer, but don't reflect that in the signature.
\end{itemize}

\subsection{Cases}

\begin{itemize}
\item each \code{case}-label must end with a \code{break}-statement.
\item A \code{switch}-statement must have a \code{default}-option, which
  handles the unknown cases, even if this cases does not exist.  This is just
  a measure for prevention of future bugs.
\item also the \code{default}-label must end with break.
\end{itemize}

\subsection{Heap}

\begin{itemize}
\item the C-library functions \code{malloc}, \code{realloc}, \code{free},
  etc. should not be used.  Use \code{new} and \code{delete} instead.
\item to free the memory of arrays use \code{delete[]}.
\item after deallocating a memory block, set the pointer to 0.
\item functions that allocate some memory must take care of freeing it at the
  end.
\item check always if a pointer is valid before you use it (except if you are
  absolutely sure that it is valid). For example:
\begin{verbatim}
myPointer=comingFromSomewhereElse();
if (notNull(myPointer)) {
    doSomething(myPointer);
} else {
    throw lti::exception("Oooh, missing pointer");
}
\end{verbatim}
\item due to the fact that pointers are the most common error sources, you
  should try to work with references instead.  You should always check after
  possible memory leaks, if you really need to use pointers.
\end{itemize}

\subsection{Text Formating}

An appropriate text formating scheme alleviates the maintenance and further
developing of existing code.  It is also helpful in order to keep the code
overview.  Following rules have shown their usefulness in practice:

\begin{itemize}
\item Use a two-spaces indentation.  Almost all modern text editors allow you
  to set the width of a tab, and if it should be replaced with spaces.  Please
  use this options.  Your code should never contain tabs, because they may
  seem to be all right for you, but for other users they will not.
\item Function names, their return type and their parameters should be in the
  same line:
\begin{verbatim}
void myFuncName(int param1, int param2);
\end{verbatim}
\item if the number of parameters is two high, and they do not fit in one
  line, try to give one parameter per line:
\begin{verbatim}
void myFuncName(const int&                 paramArraySize,
                const lti::vector<float>&  paramArray,
                lti::class&                className);
\end{verbatim}
\item all control flow primitives (\code{if}, \code{else},
\code{while}, \code{for}, \code{do}) must be followed by a complete block,
even if they consist in just one line:
\begin{verbatim}
if (pValue == 0) {
    executeFunction();
}

for (int members = 0; members[i] < maxMembers; members++) {
    // loop stops, if maxMember value reached
}
\end{verbatim}
\item the opening brace must stay exactly after the statement (and not at the
  new line).  The closing brace at the end should be at the same indentation
  level than the statement.
\item in the declaration or definition of pointer variables or references, the
  \verb+*+- or \verb+&+-operators must stay after the data type.  I.\,e.\
  \code{char* p} instead of \code{char *p}.  The reason is that C++ is a strong
  typed language, and the type of p is a pointer to \code{char}.  To declare
  more than one pointer/reference use one line per variable, or define
  previously a type with \code{typedef}.
\item before and after \code{.}, \code{->} and \code{::} there should not be
  any spaces.
\item All methods and declarations should be printable in a DIN A4-Page (\ie\
  less than 70 lines).  Lines should not be longer than 80 chars.  Source
  files should not have more than 1000 lines.
\item the priority for the application of operators should be explicitely
  given using parenthesis, in those cases where it can be ambiguous.
\end{itemize}

Some helpful tips for coding, that can also be applied to C++ can be found in
\cite{java}.

\subsection{Portability}

Many programming errors are noticed just when the code is compiled using
another compiler or another operating system.  Following tips should help
writing portable code:

\begin{itemize}
\item do not make any assumptions about the size of data types.  The
  ANSI-Standard specify only:
\begin{center}
  \code{sizeof(char)} $\leq$ \code{sizeof(short)} $\leq$ \code{sizeof(int)}
  $\leq$ \code{sizeof(long)}
\end{center}
\item Variables should never be declared as \inIndex{\code{unsigned}}, even if
  their values never become negative.  Exception to this rule is when the
  complete value range of an unsigned variable is required (for example the
  coding of RGB$\alpha$-values in a 32-bit int variable).  In these cases the
  \code{unsigned} must be explicitely given.  In those situations where
  unsigned variables of other libraries (like \product{STL}) need to interact
  with the \ltilib\, you should use the \staticcast-operator instead of the
  C-Cast method in order make assignments or comparisons.
\item a pointer does not have always the size of an integer \code{sizeof(int)}!
\item type casts between \code{signed}/\code{unsigned} must be done
  explicitely.
\item do never make assumptions about the size or location of data in memory
  (for example the number of bytes of a structure).
\item the evaluation order of the operands in an expression should not be
  assumed.  The only exceptions are boolean functions:  in \verb+a && b+ or
  \verb+a || b+ \code{a} is evaluated first.  Depending on its value, \code{b}
  is evaluated or not.
\item do not replace arithmetical operation with shift operations (this should
  be done by the compiler).  If you cannot avoid this, use comments to denote
  the semantic of the shifts.
\item avoid use of pointer arithmetic.
\item be aware of the endianness when reading or writing bytes in files.
(Big-/Little-Endian).
\item do not ever use \product{MFC} special classes, like \code{CList},
  \code{CString}, \code{CPtrList} (they are only available for \visualc).  Use
  instead standard data structures of the Standard Template Library (\stl)
  (\code{std::list}, \code{std::string}, etc.)%
\footnote{Documentation for the \stl\ can be found in \cite{stl}.}
\end{itemize}

\subsection{\inIndex{Forbidden language features}}

Following reserved words or language constructs are not allowed in the \ltilib.

\begin{itemize}
\item \verb+goto+
\item Trigraphs
\item functions without proper declaration
\item variable argument lists
\item \verb+friend+s (see also \ref{sssec:sichtbarkeit})
\item Old C header files like $<$stdlib.h$>$, $<$math.h$>$, $<$stdlib.h$>$. 
     Please note that for all this headers an equivalent C++ one exists:

     \begin{tabular}{|l|l|}
       \hline
       C ({\bf don't} use) & C++ (ok) \\
       \hline
       $<$assert.h$>$ & $<$cassert$>$ \\
       $<$ctype.h$>$ & $<$cctype$>$ \\
       $<$errno.h$>$ & $<$cerrno$>$ \\
       $<$float.h$>$ & $<$cfloat$>$ \\    
       $<$limits.h$>$ & $<$climits$>$ \\
       $<$locale.h$>$ & $<$clocale$>$ \\
       $<$math.h$>$ & $<$cmath$>$ \\
       $<$setjmp.h$>$ & $<$csetjmp$>$ \\
       $<$signal.h$>$ & $<$csignal$>$ \\
       $<$stdarg.h$>$ & $<$cstdarg$>$ \\
       $<$stdio.h$>$ & $<$cstdio$>$ \\
       $<$stdlib.h$>$ & $<$cstdlib$>$ \\
       $<$string.h$>$ & $<$cstring$>$ \\
       $<$time.h$>$ & $<$ctime$>$ \\
       \hline
     \end{tabular}
     
     You should avoid specially the use of old deprecated stream files like
     $<$fstream.h$>$, $<$iostream.h$>$, etc.  The \ltilib\ use the standard
     files (e.\,g.\ $<$fstream$>$ or $<$iostream$>$) combining them with the
     old deprecated ones leads usually to unpredictible behaviour!
\end{itemize}

\subsection{Templates}
\label{ssec:templates}
\index{template}

Templates are sort of komplex strong-typed macros.  With them you can write
code for which the types of variables or function parameters are still
unspecified.  When you use your template classes or functions you will
explicitely give the type (or types) for the variables you left ``untyped''.
Templates are extensively used for example in the \stl\ (STL).

If you use templates, you should know that
\begin{itemize}
\item no C++ compiler has 100\% support for templates.
\item templates behave sometimes like macros, with all known problems
  (annoying compiling errors, lots of code duplication and replication).
\end{itemize}

They have however as advantage, that if properly used the code can be
implemented in a very efficient manner.  Due to the speed requirements in
image processing, templates are used extensively in the \ltilib.

For an efficient and clean implementation of code using templates, the ANSI
C++ standard specifies the statement \inIndex{\code{export}}
\cite{stroustrup}.  \visualc and \product{GCC} do not support this feature.
Do not forget this when creating new functors for the \ltilib.

\subsection{Error handling}

\product{C++} provides many possibilities to deal with errors.  In the
\ltilib\ you must follow following conventions:

\subsubsection{assert}

Assertions are considered by the compiler in debug mode only.  They inform
the programmer about critical errors in code, that should \emph{NEVER} happen,
\eg\ outer bound access to matrices or other container classes.

In release versions all \code{assert} statements are ignored.  If an error
like the one mentioned before is not fixed, the program will crash without
giving any information.  Please debug your code using the debug version.  In
order to avoid crashes in bigger software systems that use the \ltilib, you
should \emph{NEVER} report usual errors in \code{apply} methods using
assertions.  For this task you should implement a defined behavior for the
functor (see below).

\subsubsection{exception}
\label{sssec:exception}

Exceptions can be used in those cases where multiple errors are possible (\eg\ during file access:
file does not exist or invalid file format).  The constructor of the class \code{lti::exception}
will expect a string with a short description of the problem.  You can of course inherit from this
class if you think it is convenient.  The most common use for exceptions is to report hardware
problems, which can require special treatment to recover, or where a specific state must be
warranted, even when error occur.

All functor classes in the \ltilib\ throw an \code{lti::exception} if the
parameters are requested but not set, or when the parameter type is wrong.

An example for the use of \code{exception}s:

\begin{verbatim}
// channels
lti::channel inputChannel, outputChannel;
// a convolution operator
lti::convolution convolver;
// initialize the convolution operator
// ... user sets the parameters here
// try to convolve
try {
  // try to convolve the inputChannel with a kernel
  // given by the user
  convolver.apply(inputChannel,outputChannel);
} catch (lti::functor::invalidParametersException& exc) {
  // some error occured with the parameters
  std::cout << "Do not forget to set the proper parameters!"
            << std::endl;
  // recover from the error
  outputChannel.clear();
} catch (lti::exception& exc) {
  // some unexpected error occured
  std::cout << "Error in convolver: " << exc.what()
            << std::endl;
  // recover from the error
  outputChannel.clear();
}

// don't know what to do with all other errors, they should be
// catched by the enclosing scope
}
\end{verbatim}

If you don't use the \code{try-catch} blocks, all thrown exceptions will
arrive enclosing scope (the calling function).

A nice introduction to exceptions can be found in \cite{stroustrup}.

\subsubsection{Defined error behavior}

This should be the error handling method of preference for all functor
\code{apply}-methods.  If two images with different sizes are given to a
functor which expects images with the same size, the functor should not crash
or throw an exception.  It just needs to return \code{false} or an invalid
result, that can be checked by the calling functions.

For all those not critical error, which are usually caused by a wrong usage of
functions and classes, please follow these rules:

\begin{itemize}
\item[On-Copy \code{apply}] The returned data objects should be reset (\eg\
  with \code{clear()} for images, matrices and vectors) or should return an
  invalid value (\eg\ the result of a distance computation, which should be
  always positive, can be in case of error negative).  These special error
  conditions must be documented.
\item[On-Place \code{apply}] The input/output parameters should not be changed.
\end{itemize}

The \code{apply}-methods return a boolean with the value \code{true} if
everything worked fine and \code{false} if a error occurred. Within the
methods of the functor class you can use \code{setStatusString()} to report
the cause of an error.  The user can check the returned boolean value if there
was a problem and get \code{getStatusString()} to determine the cause of the
problem.

\section{Creating new functors}

All functors in the \ltilib\ have a common framework.  The
\product{PERL}-Script \code{ltiGenerator} (to be found in
\code{ltilib/tools/perl/}) provides a simple way to create this class frame.
You just need to fill out some fields in the text file
\code{ltiTemplateValues.txt} and then call the script.

The expected values in \code{ltiTemplateValues.txt} are:

\begin{itemize}
\item[\code{classname}] the name of the new class.
\item[\code{parentclass}] the name of the class from which the new class will
  inherit.
\item[\code{author}] the author (or authors) of the class.
\item[\code{date}] (Optional) creation date (if leaved empty, the current date
  will be taken).
\item[\code{includes}] (Optional) A comma separated list with all header files
  required.
\item[\code{filename}] (Optional) The name of the file where this class
  should be found.  If left empty, the name will be automatically created from
  the class name, following the naming conventions of the \ltilib.
\item[\code{parameters}] The list of parameters.  Of course you can add later
  new parameters manually. You just need to update the code in the parameters
  constructor and in the methods \code{copy}, \code{read} and \code{write}.
\item[\code{applytype}] The initial parameters types for the
  \code{apply}-methods.
\end{itemize}

\section{Documentation}
\label{sec:doku}

\subsection{Documentation system}

You can find the online-documentation at
\begin{center}
  \code{http://ltilib.sourceforge.net/doc/html/index.html}
\end{center}
This will always correspond to the latest released version.  It will be
created using the comments in the header-files of each class and the file
\code{ltiDocu.h} in \code{ltilib/doc/src/}.  The documentation system
\product{Doxygen} is used for this task (for detailed information see
\cite{doxygen}).

Important is to notice that doxygen comments begin with \code{/**} and end
with \code{*/}.\index{Doxygen}

\begin{verbatim}
/**
 * This is how a comment should look like
 *
 * All doxygen commands start with \ or @ (e.g. \param).
 * Use % to avoid, that the following word gets linked.
 */
\end{verbatim}

\subsection{Documentation style}

The documentation of all classes must be written in English.  It must contain
a brief description, and a detailed explanation with all relevant information
like references to literature with theory to the implemented
algorithms.  All methods and attributes of the classes must also be
documented.  For all attributes of the parameters-classes you have to give the
used default values.

Example:

\begin{itemize}

\item Classes
\begin{verbatim}
/**
 * Computes the optical flow between two consecutive images
 * according to Horn-Schunks gradient based method. ...
 *
 * Theory: "The Computation of Optical Flow", Beauchemin
 * & Barron, ACM Computing Surveys, Vol.27 No. 3
 * Algorithm: "Computer Vision", Klette, Schluens &
 * Koschan, Springer, pp.190
 */
\end{verbatim}
\item Methods
\begin{verbatim}
/**
 * Computes the optical flow for the given sequence of images.
 * @param theSeq The sequence of images, which must have at least two elements.
 * @param mag magnitude of the optical flow
 * @param arg angle of each optical flow vector.
 * @return true if successful, false otherwise.
 */
 bool apply(const sequence<images>& theSeq,
                  channel& mag,
                  channel& arg) const;
\end{verbatim}
\item Attributes
\begin{verbatim}
/**
 * Number of iterations. iterations = 1,2,... 
 *
 * Default value: 100
 */
 int iterations;
\end{verbatim}
\end{itemize}

Of course you should also comment your source code, so that other programmers
can work on it.

%%% Local Variables:
%%% mode: latex
%%% TeX-master: "DevelopersGuide"
%%% End:
