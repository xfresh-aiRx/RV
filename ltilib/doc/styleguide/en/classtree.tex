\chapter{Class Hierarchy}
\label{chap:hierarchy}

All Objects in the \ltilib\ are in the namespace \index{\code{namespace}}
\inIndex{\code{lti}}.  The basis class in the \ltilib\ is \code{object}.

The class\code{lti::mathObject} is the parent class for many aggregate
classes, like vectors and matrices.  Most algorithms in the library operate on
this kind of data.  \code{lti::functor} objects implement the algorithms.
\code{lti::exception}-objects inherit from \code{std::exception}, which are
thrown in case of critical errors.

\begin{table}
  \caption{Basic Types in the \ltilib}
  \label{tab:generalTypes} %
  \begin{center}
    \begin{tabular}{|l|l|}
      \hline
      \code{byte}      & 8-bit Typ signed\\
      \code{ubyte}     & 8-bit Typ unsigned\\
      \code{int16}     & 16-bit Typ signed\\
      \code{uint16}    & 16-bit Typ unsigned\\
      \code{int32}     & 32-bit Typ signed\\
      \code{uint32}    & 32-bit Typ unsigned\\
      \code{point}     & 2D points with \code{lti::int32} coordinates\\
      \code{rectangle} & Two 2D points (upper-left and bottom-right)\\
      \code{rgbPixel}  & a 32-bit long RGB pixel representation\\
      \hline
    \end{tabular}
  \end{center}
\end{table}

The basic types in the \ltilib\ are listed in Table \ref{tab:generalTypes}
(all of them defined in namespace \code{lti}).

More details on these and other classes can be found in the Doxygen
documentation of the library or in \url{http://ltilib.sourceforge.net}.
Next sections provide an short overview of the documentation there%  
\footnote{%
  Please note that this section is not at all up-to-date.  Only the online
  documentation contains the actual functors, which are \emph{many} more that
  the ones listed here.  The goal of these paragraphs is only to provide a
  fast overview of the main topic of the library.
}%
.

\section{Introduction}\label{intro}

The \ltilib\ is a collection of algorithms and data structures frequently used
in the image processing.

These pages will give some informations about the usage of the \ltilib\.  For
details about the usage of each class follow the link "More..." at the header
of each class documentation.

The data structures used in the \ltilib\ are described in the section
Data Structures (\ref{dataStruct}).  All the algorithms implemented in
the \ltilib\ are Functors (\ref{functors}).

The algorithms can be organized in the following groups:
\begin{itemize}
\item
Mathematical Operations (\ref{mathOp})\par
\item
Filters (\ref{filters})\par
\item
Transforms and Modifiers (\ref{transform})\par
\item
Other image processing functors (\ref{imgproc})\par
\item
Input and Output (\ref{inout})\par
\end{itemize}

\section{Data Structures}\label{dataStruct}


The are six groups of data structures:\begin{itemize}
\item
Basic Types (\ref{basictypes})\par
\item
Vector, Matrix and Image Types (\ref{img})\par
\item
Image Regions and Contours (\ref{regions})\par
\item
Sequences (\ref{sequences})\par
\item
Pyramids (\ref{pyramids})\par
\item
Trees (\ref{trees})\par
\item
Histograms (\ref{histograms})\par
\end{itemize}
\subsection{Basic Types}\label{basictypes}


The basis types allow the representation of usual data structures like:

\begin{itemize}
\item
Color pixels: \code{lti::rgbPixel}, \code{lti::trgbPixel}$<$T$>$\item
Points in a 2D space: \code{lti::point}, \code{lti::tpoint}$<$T$>$\item
Rectangles: \code{lti::rectangle}, \code{lti::trectangle}$<$T$>$\end{itemize}

\subsection{Vector, Matrix and Image Types}\label{img}

The \ltilib\ Algorithms manipulate two types of containers: one-dimensional
(\code{lti::vector}) or two dimensional containers (\code{lti::matrix}).

Both of them are implemented as C++ templates. This allow an efficient reuse of the code with different types of data.

A vector of integers can be created with the following code:

\footnotesize\begin{verbatim}  lti::vector<int> aVct(5,0); // this creates a vector with 5 elements
                              // and initialize them with 0\end{verbatim}\normalsize


A 3x4 matrix of double (3 rows and 4 columns) will be created with

\footnotesize\begin{verbatim}  lti::matrix<double> aMat(3,4,1.0); // creates a 3x4 matrix of double
                                     // and initialized elements with 1.0\end{verbatim}\normalsize


There are some "alias" for frequently used matrix types:

We have for example the type \code{lti::image}, which is defined as a matrix
of color pixels (\code{lti::rgbPixel}).

A channel is in this library an image with monochromatic information. There are two types of channels: 8-bit channels with pixel  values between 0 and 255 (\code{lti::channel8}) and the \code{lti::channel}, which is an matrix of floats.

\subsection{Image Regions and Contours}\label{regions}


The data structures for the representation of Vector, Matrix and Image Types
(\ref{img}) are not enough to satisfy every requirement of the image
processing algorithms. Some types to allow the representations of image
regions are also required.

This function will be achieved by classes like \code{lti::pointList} and its
subclasses (\code{lti::areaPoints}, \code{lti::borderPoints} and
\code{lti::ioPoints}).

\subsection{Sequences}\label{sequences}

The \code{lti::sequence} is a sort of vector of other objects. It is indeed based on the std::vector, but with a LTI-lib standard interface.

\subsection{Histograms}\label{histograms}
The histograms are a collection of classes used to generate n-dimensional
statistics. The base class is \code{lti::histogram}. Usually a
\code{lti::mapperFunctor} is used to map the input space into the histogram
cell-space.

\subsection{Pyramids}\label{pyramids}
The multiresolutional analysis requires frequently the use of pyra\-mi\-dal
data structures. Gaussian, Laplacian and Gabor pyr\-a\-mids are examples of
this data-structures. (see \code{lti::pyramid},
\code{lti::gaussianPyramid}, \code{lti:: gaborPyramid})

\subsection{Trees}\label{trees}
 A container class for ordered n-Trees is implemented in \code{lti::tree}

\section{Functors}\label{functors}

The functors are objects which can operate on given data structures. They can
be parameterized with different values of some specific members.

For example, the functor which saves an image in a BMP-File has parameters
which include the name of the image file, the compression rate to be used, the
color depth and so on. For a Gaussian filter we need only the variance and the
dimension of the filter.

The \ltilib\ encapsulates these functor parameters in an object of an internal class called "parameters" (see \code{lti::functor::parameters}). The programmer knows, that all parameters required for a functor will be defined in its class parameters.

\subsection{Mathematical Operations}\label{mathOp}
\begin{itemize}
\item
Noise and probability distributions\begin{itemize}
\item
\code{lti::noise} adds noise with a given distribution to matrixes or vectors \item
\code{lti::randomDistribution} classes are used to generate random numbers which follow a specific
probability distribution\end{itemize}
\item
Linear Algebra\begin{itemize}
\item
\code{lti::linearEquationSystemSolutionMethod}\item
\code{lti::matrixInversion}\item
\code{lti::minimizeBasis}\item
\code{lti::principalComponentsAnalysis}\item
\code{lti::l1Distance}, \code{lti::l2Distance}\item
\code{lti::varianceFunctor} calculates variance and covariance\item
\code{lti::serialStatsFunctor} and \code{lti::meansFunctor}\end{itemize}
\end{itemize}
\subsection{Filters}\label{filters}
\begin{itemize}
\item Kernels for linear filters\begin{itemize}
\item gaussian kernels (\code{lti::gaussKernel2D}, \code{lti::gaussKernel1D})
\item gabor kernels (\code{lti::gaborKernelSep}, \code{lti::gaborKernel})
\item oriented gaussian derivatives (\code{lti::ogd1Kernel},
  \code{lti::ogd2\-Kernel})
\item gradient approximation kernels (\code{lti::gradientKernelX},
  \code{lti:: gradientKernelY})
\item binary kernels for the morphological operators
  (\code{lti::city\-Block\-Kernel}, \code{lti::chess\-Board\-Kernel},
  \code{lti::octagonal\-Kernel}, \code{lti::euclidean\-Kernel})
\end{itemize}
\item
Linear Filters (convolves objects with linear kernels)\begin{itemize}
\item
\code{lti::convolution}\item
\code{lti::squareConvolution} implements convolution with a rectangular filter kernel.\item
\code{lti::downsampling} Efficient implementation of a  filter-downsampling pair.\item
\code{lti::decimation} efficient implementation of image decimation\item
\code{lti::upsampling} Upsampling-filter pair.\item
\code{lti::filledUpsampling} efficient implementation of upsampling with a rectangular filter.\item
\code{lti::correlation}\end{itemize}
\item
Iterating functors This functors apply a simple function to all elements of the matrix. Please note, that this can also be done giving the C-Style function to the apply() method of the matrix.\begin{itemize}
\item
\code{lti::absoluteValue}\item
\code{lti::logarithm}\item
\code{lti::addScalar}, \code{lti::multiplyScalar}\item
\code{lti::square}, \code{lti::squareRoot}\end{itemize}
\end{itemize}
\begin{itemize}
\item
Morphological operators\begin{itemize}
\item
\code{lti::dilation}\item
\code{lti::erosion}\end{itemize}
\end{itemize}
\subsection{Transforms and Modifiers}\label{transform}
\begin{itemize}
\item
Color spaces transformations\begin{itemize}
\item
\code{lti::mergeImage} Merge three channels in a color image\item
\code{lti::splitImage} Splits an image in different color spaces\item
\code{lti::rgbColorQuant} color quantization in RGB space\end{itemize}
\item
Gray value transformations\begin{itemize}
\item
\code{lti::histogramEqualization}\end{itemize}
\item
Coordinate Transforms
\begin{itemize}
\item cartesian$\leftrightarrow$polar coordinate: \code{lti::cartesianToPolar} and \code{lti::
  polarToCartesian}
\end{itemize}
\item
Other Transforms\begin{itemize}
\item
FFT: \code{lti::realFFT} and \code{lti::realInvFFT}\item
Optical Flow: \code{lti::opticalFlowHS}\item
Illuminant Color Normalization: \code{lti::colorNormalization}\end{itemize}
\end{itemize}
\subsection{Other image processing functors}\label{imgproc}
\begin{itemize}
\item
Edge detectors\begin{itemize}
\item
\code{lti::susanEdges} Use SUSAN algorithm to extract the edges of an image\end{itemize}
\item
Segmentation\begin{itemize}
\item
\code{lti::thresholdSegmentation}\item
\code{lti::regionGrowing}\end{itemize}
\item
Tools for segmentation\begin{itemize}
\item
\code{lti::objectsFromMask}\item
\code{lti::boundingBox}\end{itemize}
\end{itemize}
\subsection{Input and Output}\label{inout}


\begin{itemize}
\item
The \ltilib\ supports loading and saving of Windows Bitmap files (BMP),  Jpeg files (JPEG) and Portable Network
Graphics files(PNG).
\begin{itemize}
\item
\code{lti::loadBMP}\item
\code{lti::saveBMP}\item
\code{lti::loadJPEG}\item
\code{lti::saveJPEG}\item
\code{lti::loadPNG}\item
\code{lti::savePNG}\end{itemize}
\item
It can also acquire data directly from a frame grabber:\begin{itemize}
\item \code{lti::frameGrabber}
\item \code{lti::quickCam}
\item \code{lti::toUCam}
\item \code{lti::ltiFrameGrabber}
\end{itemize}
\end{itemize}
\section{Viewer and Drawing Tools}\label{draw}
\begin{itemize}
\item
Viewers There are two viewers in the \ltilib\:\begin{itemize}
\item \code{lti::externViewer}\par This viewer
  is used to invoke in a very easy way an external application. \item
  \code{lti::viewer}\par Is a {\tt GTK}-based object,
  which runs in its own thread, to allow the user a very easy way to show
  images, fast as easy as the std::cout stream!\end{itemize}
\end{itemize}
\begin{itemize}
\item
Drawing tools\begin{itemize}
\item
\code{lti::draw}\item
\code{lti::draw3D}\item
\code{lti::epsDraw}\item
\code{lti::drawFlowField}\end{itemize}
\end{itemize}
\section{Objects for multithreading}\label{system}
 Some objects allow a OS independent use of multithreading functions:\begin{itemize}
\item
\code{lti::thread}\item
\code{lti::mutex}\item
\code{lti::semaphore}\end{itemize}
\section{Classifiers}\label{classifiers}
 It is possible with the \ltilib\ to use some classifiers. \begin{itemize}
\item \code{lti::rbf} Radial Basis Functions
\item \code{lti::MLP} Back propagation networks
\item \code{lti::svm} Support Vector Machines
\item \code{lti::hmmClassifier} Hidden Markov Models
\end{itemize}
Classifications statistics can be generated using
 \begin{itemize}
\item \code{lti::classificationStatistics}
\end{itemize}

There are many others supervised and unsupervided classifiers.  

%%% Local Variables:
%%% mode: latex
%%% TeX-master: "DevelopersGuide"
%%% End:
