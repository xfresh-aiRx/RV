\chapter{Download}
\label{chap:download}

You can get a tar ball of the \ltilib\ latest version from 

\url{http://ltilib.sourceforge.net/doc/html/index.html}

\section{Linux}

From the shell and using GNU tar you can decompress the file with
{\small
\begin{alltt}
tar xzvf ltilib.tar.gz
\end{alltt}
}

or

{\small
\begin{alltt}
tar xjvf ltilib.tar.gz
\end{alltt}
}

After that you will need to build the library:

{\small
\begin{alltt}
cd ltilib/linux
./configure
make
\end{alltt}
}

and as \code{root} or a user with write privileges on the prefix given to the
configure (usually \code{/usr/local}) you need to install the library just a

{\small
\begin{alltt}
make install
\end{alltt}
}

\section{Windows}

In \code{ltilib/win/tester} there is an example project that uses all \ltilib\
files directly.  You can make some experiments with this project.

You can of course build the library.  For this you will need a Perl-Script
interpreter installed.  In \code{ltilib/win/buildLib} execute the file
\code{buildLib.bat}.  This will create the libraries in the directory
\code{ltilib/lib}.  Additionally all header-files will be collected and placed
in \code{ltilib/lib/headerFiles}.

More information creating your own projects using the \ltilib\ can be found in
the on-line documentation.


%%% Local Variables:
%%% mode: latex
%%% TeX-master: "DevelopersGuide"
%%% End: 










