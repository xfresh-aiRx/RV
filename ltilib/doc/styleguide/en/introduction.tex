\chapter{Introduction}
\label{chap:introduction}
\pagenumbering{arabic}

The \inIndex{\ltilib}\ encloses a collection of algorithms and data structures
commonly used in image processing and computer vision applications, developed
at the Chair of Technical Computer Science (in German \emph{\textbf{L}ehrstuhl
fuer \textbf{T}echnische \textbf{I}nformatik} (\lti) ), RWTH Aachen
University.  It is written in C++ in order to allow both object oriented
programming as well as efficient resulting code.
%
Its main goal is to provide a system independent library, adhering as
closely as possible to the \cxx\ standards.  It is intended to work on
different operating systems, but it specially supports for \product{Linux/gcc}
and \product{MS WindowsNT}/\visualc\ systems.

\section{History}

In the last years, several research groups at the \lti have worked on
different applications in the field of computer vision.  
%
Before 1999, all projects using image processing algorithms (Sign Language
Recognition, Mobile Service Robots and Visual Information Retrieval) developed
their own software and applications independently.  Two typical
problems arose:

\begin{enumerate}
\item work efforts were being wasted due to unnecessary code duplication
\item reusing code was difficult or even impossible due to the lack for
  standardized interfaces.
\end{enumerate}

These reasons forced the design of an internal library, which has been now
used and extended for a while in all groups at the \lti.  The choice to
develop a new library was based on several factors:

First, we required a C++ library that followed a consistent object-oriented
concept.  This was intended to simplify the software development, enhancement
and maintainability.  Many libraries were implemented in C, or in a C-like C++
that was not appropriate to enhance or to adapt to our requirements.  Others
lacked a fundamental concept that could be employed in further developments.

Second, the library should provide not only algorithms for image processing,
but also for mathematical and statistical tasks, allowing a flexible
interchange of data between all modules.  Many existent libraries did not
fulfill these requirements.

Third, some of the regarded libraries were not maintained any more, some had
no documentation at all.  Commercial libraries were well documented and
had nifty rapid prototyping tools, but due to a ``closed source'' concept, it
would be impossible to enhance or to change existing algorithms without
reimplementing them.  This last point is unacceptable for research purposes.

Fourth, very powerful and widespread tools used in research (like
\product{Matlab}) can be used to search for solutions of relatively small
problems, but they showed to be usually too slow for more complex applications
or complete prototypes involving the whole scope from image acquisition and
processing to feature extraction and classification.  We usually required
implementations that can run in real-time or that involve huge amounts of
data.

The creation of a new library at the \lti\ was not a start-from-scratch
project, due to the existence of previous code and sufficient expertise from
all research projects.  At the beginning, an interface was specified and the
most important algorithms were adapted to it.  The \ltilib\ has grown to more
than 750 classes (more than 350000 lines), covering image processing,
mathematics, statistics, neural networks, hardware interfaces, etc. with all
objects following the same concept.

\index{advantages} The use of \emph{one} library saves development time, which
otherwise would be required in re-implementing commonly used algorithms.  This
time can now be invested in the development of new solutions or optimizing
already existing ones.  This way, not only one developers group will
benefit from improvements, but all the users of the library.
%
Its use increases also the code readability, due to the fact
that all, complex and simple tasks, will now follow general known
specifications.  Further development or maintenance are therefore easier.

The following chapters explain the concepts behind the \ltilib\ interface, and
presents all specifications required in the program coding.

%%% Local Variables: 
%%% mode: latex
%%% TeX-master: "DevelopersGuide"
%%% End: 










